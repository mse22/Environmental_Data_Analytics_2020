\PassOptionsToPackage{unicode=true}{hyperref} % options for packages loaded elsewhere
\PassOptionsToPackage{hyphens}{url}
%
\documentclass[]{article}
\usepackage{lmodern}
\usepackage{amssymb,amsmath}
\usepackage{ifxetex,ifluatex}
\usepackage{fixltx2e} % provides \textsubscript
\ifnum 0\ifxetex 1\fi\ifluatex 1\fi=0 % if pdftex
  \usepackage[T1]{fontenc}
  \usepackage[utf8]{inputenc}
  \usepackage{textcomp} % provides euro and other symbols
\else % if luatex or xelatex
  \usepackage{unicode-math}
  \defaultfontfeatures{Ligatures=TeX,Scale=MatchLowercase}
\fi
% use upquote if available, for straight quotes in verbatim environments
\IfFileExists{upquote.sty}{\usepackage{upquote}}{}
% use microtype if available
\IfFileExists{microtype.sty}{%
\usepackage[]{microtype}
\UseMicrotypeSet[protrusion]{basicmath} % disable protrusion for tt fonts
}{}
\IfFileExists{parskip.sty}{%
\usepackage{parskip}
}{% else
\setlength{\parindent}{0pt}
\setlength{\parskip}{6pt plus 2pt minus 1pt}
}
\usepackage{hyperref}
\hypersetup{
            pdftitle={Assignment 3: Data Exploration},
            pdfauthor={Monisha Eadala},
            pdfborder={0 0 0},
            breaklinks=true}
\urlstyle{same}  % don't use monospace font for urls
\usepackage[margin=2.54cm]{geometry}
\usepackage{color}
\usepackage{fancyvrb}
\newcommand{\VerbBar}{|}
\newcommand{\VERB}{\Verb[commandchars=\\\{\}]}
\DefineVerbatimEnvironment{Highlighting}{Verbatim}{commandchars=\\\{\}}
% Add ',fontsize=\small' for more characters per line
\usepackage{framed}
\definecolor{shadecolor}{RGB}{248,248,248}
\newenvironment{Shaded}{\begin{snugshade}}{\end{snugshade}}
\newcommand{\AlertTok}[1]{\textcolor[rgb]{0.94,0.16,0.16}{#1}}
\newcommand{\AnnotationTok}[1]{\textcolor[rgb]{0.56,0.35,0.01}{\textbf{\textit{#1}}}}
\newcommand{\AttributeTok}[1]{\textcolor[rgb]{0.77,0.63,0.00}{#1}}
\newcommand{\BaseNTok}[1]{\textcolor[rgb]{0.00,0.00,0.81}{#1}}
\newcommand{\BuiltInTok}[1]{#1}
\newcommand{\CharTok}[1]{\textcolor[rgb]{0.31,0.60,0.02}{#1}}
\newcommand{\CommentTok}[1]{\textcolor[rgb]{0.56,0.35,0.01}{\textit{#1}}}
\newcommand{\CommentVarTok}[1]{\textcolor[rgb]{0.56,0.35,0.01}{\textbf{\textit{#1}}}}
\newcommand{\ConstantTok}[1]{\textcolor[rgb]{0.00,0.00,0.00}{#1}}
\newcommand{\ControlFlowTok}[1]{\textcolor[rgb]{0.13,0.29,0.53}{\textbf{#1}}}
\newcommand{\DataTypeTok}[1]{\textcolor[rgb]{0.13,0.29,0.53}{#1}}
\newcommand{\DecValTok}[1]{\textcolor[rgb]{0.00,0.00,0.81}{#1}}
\newcommand{\DocumentationTok}[1]{\textcolor[rgb]{0.56,0.35,0.01}{\textbf{\textit{#1}}}}
\newcommand{\ErrorTok}[1]{\textcolor[rgb]{0.64,0.00,0.00}{\textbf{#1}}}
\newcommand{\ExtensionTok}[1]{#1}
\newcommand{\FloatTok}[1]{\textcolor[rgb]{0.00,0.00,0.81}{#1}}
\newcommand{\FunctionTok}[1]{\textcolor[rgb]{0.00,0.00,0.00}{#1}}
\newcommand{\ImportTok}[1]{#1}
\newcommand{\InformationTok}[1]{\textcolor[rgb]{0.56,0.35,0.01}{\textbf{\textit{#1}}}}
\newcommand{\KeywordTok}[1]{\textcolor[rgb]{0.13,0.29,0.53}{\textbf{#1}}}
\newcommand{\NormalTok}[1]{#1}
\newcommand{\OperatorTok}[1]{\textcolor[rgb]{0.81,0.36,0.00}{\textbf{#1}}}
\newcommand{\OtherTok}[1]{\textcolor[rgb]{0.56,0.35,0.01}{#1}}
\newcommand{\PreprocessorTok}[1]{\textcolor[rgb]{0.56,0.35,0.01}{\textit{#1}}}
\newcommand{\RegionMarkerTok}[1]{#1}
\newcommand{\SpecialCharTok}[1]{\textcolor[rgb]{0.00,0.00,0.00}{#1}}
\newcommand{\SpecialStringTok}[1]{\textcolor[rgb]{0.31,0.60,0.02}{#1}}
\newcommand{\StringTok}[1]{\textcolor[rgb]{0.31,0.60,0.02}{#1}}
\newcommand{\VariableTok}[1]{\textcolor[rgb]{0.00,0.00,0.00}{#1}}
\newcommand{\VerbatimStringTok}[1]{\textcolor[rgb]{0.31,0.60,0.02}{#1}}
\newcommand{\WarningTok}[1]{\textcolor[rgb]{0.56,0.35,0.01}{\textbf{\textit{#1}}}}
\usepackage{graphicx,grffile}
\makeatletter
\def\maxwidth{\ifdim\Gin@nat@width>\linewidth\linewidth\else\Gin@nat@width\fi}
\def\maxheight{\ifdim\Gin@nat@height>\textheight\textheight\else\Gin@nat@height\fi}
\makeatother
% Scale images if necessary, so that they will not overflow the page
% margins by default, and it is still possible to overwrite the defaults
% using explicit options in \includegraphics[width, height, ...]{}
\setkeys{Gin}{width=\maxwidth,height=\maxheight,keepaspectratio}
\setlength{\emergencystretch}{3em}  % prevent overfull lines
\providecommand{\tightlist}{%
  \setlength{\itemsep}{0pt}\setlength{\parskip}{0pt}}
\setcounter{secnumdepth}{0}
% Redefines (sub)paragraphs to behave more like sections
\ifx\paragraph\undefined\else
\let\oldparagraph\paragraph
\renewcommand{\paragraph}[1]{\oldparagraph{#1}\mbox{}}
\fi
\ifx\subparagraph\undefined\else
\let\oldsubparagraph\subparagraph
\renewcommand{\subparagraph}[1]{\oldsubparagraph{#1}\mbox{}}
\fi

% set default figure placement to htbp
\makeatletter
\def\fps@figure{htbp}
\makeatother


\title{Assignment 3: Data Exploration}
\author{Monisha Eadala}
\date{}

\begin{document}
\maketitle

\hypertarget{overview}{%
\subsection{OVERVIEW}\label{overview}}

This exercise accompanies the lessons in Environmental Data Analytics on
Data Exploration.

\hypertarget{directions}{%
\subsection{Directions}\label{directions}}

\begin{enumerate}
\def\labelenumi{\arabic{enumi}.}
\tightlist
\item
  Change ``Student Name'' on line 3 (above) with your name.
\item
  Work through the steps, \textbf{creating code and output} that fulfill
  each instruction.
\item
  Be sure to \textbf{answer the questions} in this assignment document.
\item
  When you have completed the assignment, \textbf{Knit} the text and
  code into a single PDF file.
\item
  After Knitting, submit the completed exercise (PDF file) to the
  dropbox in Sakai. Add your last name into the file name (e.g.,
  ``Salk\_A03\_DataExploration.Rmd'') prior to submission.
\end{enumerate}

The completed exercise is due on Tuesday, January 28 at 1:00 pm.

\hypertarget{set-up-your-r-session}{%
\subsection{Set up your R session}\label{set-up-your-r-session}}

\begin{enumerate}
\def\labelenumi{\arabic{enumi}.}
\tightlist
\item
  Check your working directory, load necessary packages (tidyverse), and
  upload two datasets: the ECOTOX neonicotinoid dataset
  (ECOTOX\_Neonicotinoids\_Insects\_raw.csv) and the Niwot Ridge NEON
  dataset for litter and woody debris
  (NEON\_NIWO\_Litter\_massdata\_2018-08\_raw.csv). Name these datasets
  ``Neonics'' and ``Litter'', respectively.
\end{enumerate}

\begin{Shaded}
\begin{Highlighting}[]
\CommentTok{# To set up working directory}
\KeywordTok{getwd}\NormalTok{()}
\end{Highlighting}
\end{Shaded}

\begin{verbatim}
## [1] "/Users/monishaeadala/Environmental_Data_Analytics_2020"
\end{verbatim}

\begin{Shaded}
\begin{Highlighting}[]
\CommentTok{# To load packages}
\KeywordTok{library}\NormalTok{(tidyverse)}

\KeywordTok{options}\NormalTok{(}\DataTypeTok{tinytex.verbose =} \OtherTok{TRUE}\NormalTok{)}

\CommentTok{# To import datasets}
\NormalTok{Neonics <-}\StringTok{ }\KeywordTok{read.csv}\NormalTok{(}\StringTok{"./Data/Raw/ECOTOX_Neonicotinoids_Insects_raw.csv"}\NormalTok{)}
\NormalTok{Litter <-}\StringTok{ }\KeywordTok{read.csv}\NormalTok{(}\StringTok{"./Data/Raw/NEON_NIWO_Litter_massdata_2018-08_raw.csv"}\NormalTok{)}
\end{Highlighting}
\end{Shaded}

\hypertarget{learn-about-your-system}{%
\subsection{Learn about your system}\label{learn-about-your-system}}

\begin{enumerate}
\def\labelenumi{\arabic{enumi}.}
\setcounter{enumi}{1}
\tightlist
\item
  The neonicotinoid dataset was collected from the Environmental
  Protection Agency's ECOTOX Knowledgebase, a database for ecotoxicology
  research. Neonicotinoids are a class of insecticides used widely in
  agriculture. The dataset that has been pulled includes all studies
  published on insects. Why might we be interested in the ecotoxicologoy
  of neonicotinoids on insects? Feel free to do a brief internet search
  if you feel you need more background information.
\end{enumerate}

\begin{quote}
Answer: We are interested in the ecotoxicologoy of neonicotinoids on
insects since we want to know the effects of neonicotinoids on both
target organisms and non-target organisms. From glancing at the Neonics
dataset, we can tell that neonicotinoids seem to be effecting some of
the beneficial insects and the harmful pests in a similar fashion (by
causing mortality). For example, even though they seem to be causing
mortality to the target pests such as coffee bean weevils and
sweetpotato whiteflies, they are also causing the same harm to
beneficial insects such as honey bees, bumble bees and few of the lady
beetle species. Also it is necessary to have and understand such data
that could aid in controlling the dossage and concentration of the
pesticide according to the level of pest control required without going
overboard causing harm to the ecological balance.
\end{quote}

\begin{enumerate}
\def\labelenumi{\arabic{enumi}.}
\setcounter{enumi}{2}
\tightlist
\item
  The Niwot Ridge litter and woody debris dataset was collected from the
  National Ecological Observatory Network, which collectively includes
  81 aquatic and terrestrial sites across 20 ecoclimatic domains. 32 of
  these sites sample forest litter and woody debris, and we will focus
  on the Niwot Ridge long-term ecological research (LTER) station in
  Colorado. Why might we be interested in studying litter and woody
  debris that falls to the ground in forests? Feel free to do a brief
  internet search if you feel you need more background information.
\end{enumerate}

\begin{quote}
Answer: The litter and woody debris data may be used to estimate the
annual Aboveground Net Primary Productivity (ANPP) and aboveground
biomass at plot, site, and continental scales that are important to
understand vegetative carbon fluxes over the time and answer very
important questions regarding the global carbon balance, the location of
the missing carbon sink, and predictions of global climate change.
\end{quote}

\begin{enumerate}
\def\labelenumi{\arabic{enumi}.}
\setcounter{enumi}{3}
\tightlist
\item
  How is litter and woody debris sampled as part of the NEON network?
  Read the NEON\_Litterfall\_UserGuide.pdf document to learn more. List
  three pieces of salient information about the sampling methods here:
\end{enumerate}

\begin{quote}
Answer: Litter and fine woody debris are collected individually from two
different traps - elevated and ground traps. All masses reported after
processing are reported at the spatial resolution of a single trap and
the temporal resolution of a single collection event. * Spatial
Sampling: For elevated traps, litter and fine woody debris sampling is
executed at terrestrial NEON sites that contain woody vegetation more
than 2m tall. Locations of tower plots are selected randomly within the
90\% flux footprint of the primary and secondary airsheds. * In sites
with forested tower airsheds, the litter sampling is targeted to take
place in 20 40m x 40m plots. In sites with low-statured vegetation over
the tower airsheds,litter sampling is targeted to take place in 4 40m x
40m tower plots (to accommodate co-located soil sampling) plus 26 20m x
20m plots. Also, Trap placement within plots may be either targeted or
randomized, depending on the vegetation. In sites with more than 50\%
aerial cover of woody vegetation more than 2m in height, placement of
litter traps is random. * Temporal Sampling: Ground traps are sampled
once per year. Target sampling frequency for elevated traps varies by
vegetation present at the site, with frequent sampling (1x every 2weeks)
in deciduous forest sites during senescence, and infrequent year-round
sampling (1x every 1-2 months) at evergreen sites.
\end{quote}

\hypertarget{obtain-basic-summaries-of-your-data-neonics}{%
\subsection{Obtain basic summaries of your data
(Neonics)}\label{obtain-basic-summaries-of-your-data-neonics}}

\begin{enumerate}
\def\labelenumi{\arabic{enumi}.}
\setcounter{enumi}{4}
\tightlist
\item
  What are the dimensions of the dataset?
\end{enumerate}

\begin{Shaded}
\begin{Highlighting}[]
\KeywordTok{dim}\NormalTok{(Neonics) }\CommentTok{# Gives the dimensions (number of rows and columns) of the dataset (in this the number of rows are 4623, while the number of columns are 30)}
\end{Highlighting}
\end{Shaded}

\begin{verbatim}
## [1] 4623   30
\end{verbatim}

\begin{enumerate}
\def\labelenumi{\arabic{enumi}.}
\setcounter{enumi}{5}
\tightlist
\item
  Using the \texttt{summary} function, determine the most common effects
  that are studied. Why might these effects specifically be of interest?
\end{enumerate}

\begin{Shaded}
\begin{Highlighting}[]
\KeywordTok{summary}\NormalTok{(Neonics}\OperatorTok{$}\NormalTok{Effect) }\CommentTok{# Lists all the effects with their respective frequency; the ones with the most frequency are the most common. }
\end{Highlighting}
\end{Shaded}

\begin{verbatim}
##     Accumulation        Avoidance         Behavior     Biochemistry 
##               12              102              360               11 
##          Cell(s)      Development        Enzyme(s) Feeding behavior 
##                9              136               62              255 
##         Genetics           Growth        Histology       Hormone(s) 
##               82               38                5                1 
##    Immunological     Intoxication       Morphology        Mortality 
##               16               12               22             1493 
##       Physiology       Population     Reproduction 
##                7             1803              197
\end{verbatim}

\begin{quote}
Answer: The most common effects that are studied are ``population'',
``mortality'', ``behavior'', ``feeding behavior'', ``reproduction'', and
``development''. These effects might specifically be of interest to
understand what are some of the most common effects of the insecticides
and to even determine if such effects are detrimental to the
environment. Such understanding can form a basis for environmental
decision analysis and can determine the future use of such insecticides.
\end{quote}

\begin{enumerate}
\def\labelenumi{\arabic{enumi}.}
\setcounter{enumi}{6}
\tightlist
\item
  Using the \texttt{summary} function, determine the six most commonly
  studied species in the dataset (common name). What do these species
  have in common, and why might they be of interest over other insects?
  Feel free to do a brief internet search for more information if
  needed.
\end{enumerate}

\begin{Shaded}
\begin{Highlighting}[]
\KeywordTok{summary}\NormalTok{(Neonics}\OperatorTok{$}\NormalTok{Species.Common.Name) }\CommentTok{# lists the studied species with the most common ones listed at the top}
\end{Highlighting}
\end{Shaded}

\begin{verbatim}
##                          Honey Bee                     Parasitic Wasp 
##                                667                                285 
##              Buff Tailed Bumblebee                Carniolan Honey Bee 
##                                183                                152 
##                         Bumble Bee                   Italian Honeybee 
##                                140                                113 
##                    Japanese Beetle                  Asian Lady Beetle 
##                                 94                                 76 
##                     Euonymus Scale                           Wireworm 
##                                 75                                 69 
##                  European Dark Bee                  Minute Pirate Bug 
##                                 66                                 62 
##               Asian Citrus Psyllid                      Parastic Wasp 
##                                 60                                 58 
##             Colorado Potato Beetle                    Parasitoid Wasp 
##                                 57                                 51 
##                Erythrina Gall Wasp                       Beetle Order 
##                                 49                                 47 
##        Snout Beetle Family, Weevil           Sevenspotted Lady Beetle 
##                                 47                                 46 
##                     True Bug Order              Buff-tailed Bumblebee 
##                                 45                                 39 
##                       Aphid Family                     Cabbage Looper 
##                                 38                                 38 
##               Sweetpotato Whitefly                      Braconid Wasp 
##                                 37                                 33 
##                       Cotton Aphid                     Predatory Mite 
##                                 33                                 33 
##             Ladybird Beetle Family                         Parasitoid 
##                                 30                                 30 
##                      Scarab Beetle                      Spring Tiphia 
##                                 29                                 29 
##                        Thrip Order               Ground Beetle Family 
##                                 29                                 27 
##                 Rove Beetle Family                      Tobacco Aphid 
##                                 27                                 27 
##                       Chalcid Wasp             Convergent Lady Beetle 
##                                 25                                 25 
##                      Stingless Bee                  Spider/Mite Class 
##                                 25                                 24 
##                Tobacco Flea Beetle                   Citrus Leafminer 
##                                 24                                 23 
##                    Ladybird Beetle                          Mason Bee 
##                                 23                                 22 
##                           Mosquito                      Argentine Ant 
##                                 22                                 21 
##                             Beetle         Flatheaded Appletree Borer 
##                                 21                                 20 
##               Horned Oak Gall Wasp                 Leaf Beetle Family 
##                                 20                                 20 
##                  Potato Leafhopper         Tooth-necked Fungus Beetle 
##                                 20                                 20 
##                       Codling Moth          Black-spotted Lady Beetle 
##                                 19                                 18 
##                       Calico Scale                Fairyfly Parasitoid 
##                                 18                                 18 
##                        Lady Beetle             Minute Parasitic Wasps 
##                                 18                                 18 
##                          Mirid Bug                   Mulberry Pyralid 
##                                 18                                 18 
##                           Silkworm                     Vedalia Beetle 
##                                 18                                 18 
##              Araneoid Spider Order                          Bee Order 
##                                 17                                 17 
##                     Egg Parasitoid                       Insect Class 
##                                 17                                 17 
##           Moth And Butterfly Order       Oystershell Scale Parasitoid 
##                                 17                                 17 
## Hemlock Woolly Adelgid Lady Beetle              Hemlock Wooly Adelgid 
##                                 16                                 16 
##                               Mite                        Onion Thrip 
##                                 16                                 16 
##              Western Flower Thrips                       Corn Earworm 
##                                 15                                 14 
##                  Green Peach Aphid                          House Fly 
##                                 14                                 14 
##                          Ox Beetle                 Red Scale Parasite 
##                                 14                                 14 
##                 Spined Soldier Bug              Armoured Scale Family 
##                                 14                                 13 
##                   Diamondback Moth                      Eulophid Wasp 
##                                 13                                 13 
##                  Monarch Butterfly                      Predatory Bug 
##                                 13                                 13 
##              Yellow Fever Mosquito                Braconid Parasitoid 
##                                 13                                 12 
##                       Common Thrip       Eastern Subterranean Termite 
##                                 12                                 12 
##                             Jassid                         Mite Order 
##                                 12                                 12 
##                          Pea Aphid                   Pond Wolf Spider 
##                                 12                                 12 
##           Spotless Ladybird Beetle             Glasshouse Potato Wasp 
##                                 11                                 10 
##                           Lacewing            Southern House Mosquito 
##                                 10                                 10 
##            Two Spotted Lady Beetle                         Ant Family 
##                                 10                                  9 
##                       Apple Maggot                            (Other) 
##                                  9                                670
\end{verbatim}

\begin{quote}
Answer: The six most commonly studies species in the dataset are ``honey
bee'', ``parasitic wasp'', ``buff tailed bumblebee'', ``Carniolan honey
bee'', and ``bumble bee'' and ``Italian honeybee''. the one thing all
these species have in common is that they are all considered beneficial
insects that feed on other pests that harm the crops. They all have good
reputation in the field of garden and crop management/ pest control.
\end{quote}

\begin{enumerate}
\def\labelenumi{\arabic{enumi}.}
\setcounter{enumi}{7}
\tightlist
\item
  Concentrations are always a numeric value. What is the class of
  Conc.1..Author. in the dataset, and why is it not numeric?
\end{enumerate}

\begin{Shaded}
\begin{Highlighting}[]
\KeywordTok{class}\NormalTok{(Neonics}\OperatorTok{$}\NormalTok{Conc.}\DecValTok{1}\NormalTok{..Author.)}
\end{Highlighting}
\end{Shaded}

\begin{verbatim}
## [1] "factor"
\end{verbatim}

\begin{quote}
Answer: The class of Conc.1..Author. is ``factor'' and not numeric, and
this is because the values cannot be considered numeric yet given that
they all have different units. Only after converting all the units to a
single consistent unit, we should convert the class of the variable to
numeric.
\end{quote}

\hypertarget{explore-your-data-graphically-neonics}{%
\subsection{Explore your data graphically
(Neonics)}\label{explore-your-data-graphically-neonics}}

\begin{enumerate}
\def\labelenumi{\arabic{enumi}.}
\setcounter{enumi}{8}
\tightlist
\item
  Using \texttt{geom\_freqpoly}, generate a plot of the number of
  studies conducted by publication year.
\end{enumerate}

\begin{Shaded}
\begin{Highlighting}[]
\KeywordTok{ggplot}\NormalTok{(Neonics) }\OperatorTok{+}
\StringTok{  }\KeywordTok{geom_freqpoly}\NormalTok{(}\KeywordTok{aes}\NormalTok{(}\DataTypeTok{x =}\NormalTok{ Publication.Year), }\DataTypeTok{bins =} \DecValTok{50}\NormalTok{) }\CommentTok{# Generates a plot of the number of studies conducted by publication year}
\end{Highlighting}
\end{Shaded}

\includegraphics{A03_DataExploration_files/figure-latex/unnamed-chunk-6-1.pdf}

\begin{enumerate}
\def\labelenumi{\arabic{enumi}.}
\setcounter{enumi}{9}
\tightlist
\item
  Reproduce the same graph but now add a color aesthetic so that
  different Test.Location are displayed as different colors.
\end{enumerate}

\begin{Shaded}
\begin{Highlighting}[]
\KeywordTok{ggplot}\NormalTok{(Neonics) }\OperatorTok{+}
\StringTok{  }\KeywordTok{geom_freqpoly}\NormalTok{(}\KeywordTok{aes}\NormalTok{(}\DataTypeTok{x =}\NormalTok{ Publication.Year, }\DataTypeTok{color =}\NormalTok{ Test.Location), }\DataTypeTok{bins =} \DecValTok{50}\NormalTok{) }\CommentTok{# Reproduces the same graph but now add a color aesthetic so that different Test.Location are displayed as different colors}
\end{Highlighting}
\end{Shaded}

\includegraphics{A03_DataExploration_files/figure-latex/unnamed-chunk-7-1.pdf}

Interpret this graph. What are the most common test locations, and do
they differ over time?

\begin{quote}
Answer: The most common test locations overall according to the graph
seem to be Lab, followed by Field natural, Field artifical and then
Field underdeterminable. However, they do differ over the time. For
instance, right before 2010 (around 2009) Field natural seems to be more
common, while right after 2010 Lab seems to be more common than Field
natural.
\end{quote}

\begin{enumerate}
\def\labelenumi{\arabic{enumi}.}
\setcounter{enumi}{10}
\tightlist
\item
  Create a bar graph of Endpoint counts. What are the two most common
  end points, and how are they defined? Consult the ECOTOX\_CodeAppendix
  for more information.
\end{enumerate}

\begin{Shaded}
\begin{Highlighting}[]
\KeywordTok{ggplot}\NormalTok{(Neonics, }\KeywordTok{aes}\NormalTok{(}\DataTypeTok{x =}\NormalTok{ Endpoint)) }\OperatorTok{+}
\StringTok{  }\KeywordTok{geom_bar}\NormalTok{() }\CommentTok{# Creates a bar graph of Endpoint counts}
\end{Highlighting}
\end{Shaded}

\includegraphics{A03_DataExploration_files/figure-latex/unnamed-chunk-8-1.pdf}

\begin{quote}
Answer: The two most common endpoints are NOEL (most common) and LOEL
(second-most common). NOEL is the abbreviation for
No-observable-effect-level, and it is defined as the highest dose
(concentration) producing effects not significantly different from
responses of controls according to author's reported statistical test
(NOEAL/NOEC). LOEL is the abbreviation for
Lowest-observable-effect-level, and it is definied as the lowest dose
(concentration) producing effects that were significantly different (as
reported by authors) from responses of controls (LOEAL/LOEC).
\end{quote}

\hypertarget{explore-your-data-litter}{%
\subsection{Explore your data (Litter)}\label{explore-your-data-litter}}

\begin{enumerate}
\def\labelenumi{\arabic{enumi}.}
\setcounter{enumi}{11}
\tightlist
\item
  Determine the class of collectDate. Is it a date? If not, change to a
  date and confirm the new class of the variable. Using the
  \texttt{unique} function, determine which dates litter was sampled in
  August 2018.
\end{enumerate}

\begin{Shaded}
\begin{Highlighting}[]
\KeywordTok{class}\NormalTok{(Litter}\OperatorTok{$}\NormalTok{collectDate) }\CommentTok{#determines the class of collectDate (in this case it is a "factor")}
\end{Highlighting}
\end{Shaded}

\begin{verbatim}
## [1] "factor"
\end{verbatim}

\begin{Shaded}
\begin{Highlighting}[]
\NormalTok{Litter}\OperatorTok{$}\NormalTok{collectDate <-}\StringTok{ }\KeywordTok{as.Date}\NormalTok{(Litter}\OperatorTok{$}\NormalTok{collectDate, }\DataTypeTok{format =} \StringTok{"%Y-%m-%d"}\NormalTok{) }\CommentTok{# Changes it to a date}
\KeywordTok{class}\NormalTok{(Litter}\OperatorTok{$}\NormalTok{collectDate) }\CommentTok{# confirmed that the new class of collectDate is a "Date"}
\end{Highlighting}
\end{Shaded}

\begin{verbatim}
## [1] "Date"
\end{verbatim}

\begin{enumerate}
\def\labelenumi{\arabic{enumi}.}
\setcounter{enumi}{12}
\tightlist
\item
  Using the \texttt{unique} function, determine how many plots were
  sampled at Niwot Ridge. How is the information obtained from
  \texttt{unique} different from that obtained from \texttt{summary}?
\end{enumerate}

\begin{Shaded}
\begin{Highlighting}[]
\KeywordTok{unique}\NormalTok{(Litter}\OperatorTok{$}\NormalTok{plotID) }\CommentTok{# Determines how many plots were sampled at Niwot Ridge without dublication}
\end{Highlighting}
\end{Shaded}

\begin{verbatim}
##  [1] NIWO_061 NIWO_064 NIWO_067 NIWO_040 NIWO_041 NIWO_063 NIWO_047 NIWO_051
##  [9] NIWO_058 NIWO_046 NIWO_062 NIWO_057
## 12 Levels: NIWO_040 NIWO_041 NIWO_046 NIWO_047 NIWO_051 NIWO_057 ... NIWO_067
\end{verbatim}

\begin{quote}
Answer: The `unique' function eliminates duplicate elements/rows from a
vector, data frame or array, while the `summary' function does not. In
other words, the `unique' function strictly returns the unique rows in a
dataframe while the summary returns all the rows in a dataframe. For the
`unique' function, all the elements in each row must match in order to
be termed as duplicates.
\end{quote}

\begin{enumerate}
\def\labelenumi{\arabic{enumi}.}
\setcounter{enumi}{13}
\tightlist
\item
  Create a bar graph of functionalGroup counts. This shows you what type
  of litter is collected at the Niwot Ridge sites. Notice that litter
  types are fairly equally distributed across the Niwot Ridge sites.
\end{enumerate}

\begin{Shaded}
\begin{Highlighting}[]
\KeywordTok{ggplot}\NormalTok{(Litter, }\KeywordTok{aes}\NormalTok{(}\DataTypeTok{x =}\NormalTok{ functionalGroup)) }\OperatorTok{+}
\StringTok{  }\KeywordTok{geom_bar}\NormalTok{() }\CommentTok{# Creates a bar graph of functionalGroup counts}
\end{Highlighting}
\end{Shaded}

\includegraphics{A03_DataExploration_files/figure-latex/unnamed-chunk-11-1.pdf}

\begin{enumerate}
\def\labelenumi{\arabic{enumi}.}
\setcounter{enumi}{14}
\tightlist
\item
  Using \texttt{geom\_boxplot} and \texttt{geom\_violin}, create a
  boxplot and a violin plot of dryMass by functionalGroup.
\end{enumerate}

\begin{Shaded}
\begin{Highlighting}[]
\KeywordTok{ggplot}\NormalTok{(Litter) }\OperatorTok{+}
\StringTok{  }\KeywordTok{geom_boxplot}\NormalTok{(}\KeywordTok{aes}\NormalTok{(}\DataTypeTok{x =}\NormalTok{ functionalGroup, }\DataTypeTok{y =}\NormalTok{ dryMass)) }\CommentTok{# Creates a boxplot of dryMass by functionalGroup}
\end{Highlighting}
\end{Shaded}

\includegraphics{A03_DataExploration_files/figure-latex/unnamed-chunk-12-1.pdf}

\begin{Shaded}
\begin{Highlighting}[]
\KeywordTok{ggplot}\NormalTok{(Litter) }\OperatorTok{+}
\StringTok{  }\KeywordTok{geom_violin}\NormalTok{(}\KeywordTok{aes}\NormalTok{(}\DataTypeTok{x =}\NormalTok{ functionalGroup, }\DataTypeTok{y =}\NormalTok{ dryMass), }
              \DataTypeTok{draw_quantiles =} \KeywordTok{c}\NormalTok{(}\FloatTok{0.25}\NormalTok{, }\FloatTok{0.5}\NormalTok{, }\FloatTok{0.75}\NormalTok{)) }\CommentTok{# Creates a violin plot of dryMass by functionalGroup}
\end{Highlighting}
\end{Shaded}

\begin{verbatim}
## Warning in regularize.values(x, y, ties, missing(ties)): collapsing to unique
## 'x' values

## Warning in regularize.values(x, y, ties, missing(ties)): collapsing to unique
## 'x' values

## Warning in regularize.values(x, y, ties, missing(ties)): collapsing to unique
## 'x' values
\end{verbatim}

\includegraphics{A03_DataExploration_files/figure-latex/unnamed-chunk-12-2.pdf}

Why is the boxplot a more effective visualization option than the violin
plot in this case?

\begin{quote}
Answer: A boxplot is a more effective visualization option than the
violin plot in this case because the boxplot is giving us more
information; especially regarding the median, first quartile, third
quartile and outliers. On the other hand, the only information the
violin plot is giving us is the minimum and the maximum observations
(which the box plot gives us as well). Also, R tells us that there are 3
counts of collapsing due to unique `x' values in the violin plot, which
makes the visual even more unappealing. Therefore, a boxplot is much
better in this case.
\end{quote}

What type(s) of litter tend to have the highest biomass at these sites?

\begin{quote}
Answer: From the box plot it is evident that Needles have the highest
biomass at these sites.
\end{quote}

\end{document}
